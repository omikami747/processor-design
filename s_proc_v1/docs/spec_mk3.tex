% Created 2022-10-13 Thu 23:01
% Intended LaTeX compiler: pdflatex
\documentclass[11pt]{article}
\usepackage[utf8]{inputenc}
\usepackage[T1]{fontenc}
\usepackage{graphicx}
\usepackage{grffile}
\usepackage{longtable}
\usepackage{wrapfig}
\usepackage{rotating}
\usepackage[normalem]{ulem}
\usepackage{amsmath}
\usepackage{textcomp}
\usepackage{amssymb}
\usepackage{capt-of}
\usepackage{hyperref}
\usepackage{parskip}
\author{Omkar Girish Kamath}
\date{\today}
\title{Specification Document}
\hypersetup{
  pdfauthor={Omkar Girish Kamath},
  pdftitle={Specification Document},
  pdfkeywords={},
  pdfsubject={},
  pdfcreator={Emacs 26.3 (Org mode 9.1.9)}, 
  pdflang={English}}
\begin{document}

\maketitle
\tableofcontents

\section{Processor}
\subsection{Instruction Set}
\label{sec:org4f08230}

In this instruction syntax X=Not used, K=Constant, A=Instruction Address, P=Data Address

\begin{longtable}{|l|l|l|}
  \caption{Instruction Set of the Simple CPU}\\ \hline
          {\bf Opcode} & {\bf Instruction} & {\bf RTL} \\ 
          \endfirsthead
          \caption[]{(\em continued from previous page)}\\
          \hline
              {\bf Opcode} & {\bf Instruction} & {\bf RTL} \\ \hline 
              \endhead
              \multicolumn{3}{|r|}{{\em continued on next page}} \\ \hline
              \endfoot
              \endlastfoot
              \hline \hline
              \texttt{Load ACC kk} &  \texttt{0000 XXXX KKKKKKKK} & \texttt{ACC <- KK} \\ \hline
              
              \texttt{Add ACC kk} &  \texttt{0100 XXXX KKKKKKKK} & \texttt{ACC <- ACC + KK} \\ \hline
              
              \texttt{And ACC kk} &  \texttt{0001 XXXX KKKKKKKK} &  \texttt{ACC <- ACC \& KK} \\ \hline
              
              \texttt{Sub ACC kk} &  \texttt{0110 XXXX KKKKKKKK} & \texttt{ACC <- ACC - KK} \\ \hline
              
              \texttt{Input ACC pp} &  \texttt{1010 XXXX PPPPPPPP} & \texttt{ACC <- M[PP]} \\ \hline
              
              \texttt{Output ACC pp} &  \texttt{1110 XXXX PPPPPPPP} & \texttt{M[PP] <- ACC} \\ \hline
              
              \texttt{Jump U aa} &  \texttt{1000 XXXX AAAAAAAA} & \texttt{PC <- AA} \\ \hline
              
              \texttt{Jump Z aa} &  \texttt{1001 00XX AAAAAAAA} & \texttt{IF Z=1 PC <- AA ELSE PC <- PC + 1} \\ \hline
              
              \texttt{Jump C aa} &  \texttt{1001 10XX AAAAAAAA} & \texttt{IF C=1 PC <- AA ELSE PC <- PC + 1} \\ \hline
              
              \texttt{Jump NZ aa} &  \texttt{1001 01XX AAAAAAAA} & \texttt{IF Z=0 PC <- AA ELSE PC <- PC + 1} \\ \hline
              
              \texttt{Jump NC aa} &  \texttt{1001 11XX AAAAAAAA} & \texttt{IF C=0 PC <- AA ELSE PC <- PC + 1}\\ \hline
              
\end{longtable}


Here '->' indicates updated with .

The processor has an extra cycle to save on hardware which would have been required for incrementing the PC . So the processor follows \textbf{fetch-decode-execute-increment} cycle .   

\subsection{Input Output Interface}
\begin{table}[h]
  \begin{center}
    \caption{I/O signals}  
    \vspace*{5mm}
    \begin{tabular}{||l|l|l||l||}
      \hline
          {\bf Signals} & { \bf type } & {\bf size} & {\bf description}\\ \hline
          {\bf } & { \bf  } & {\bf } & {\bf } \\ \hline
          {\bf } & { \bf  } & {\bf } & {\bf } \\ \hline
    \end{tabular}
  \end{center}
\end{table}
\section{Memory}
\subsection{Description}
           {\bf Size of RAM -> 4 Kilobytes} \\
           RAM used is DDR5 Synchronous Dynamic RAM . \\
           Instruction length is \textit{16 bit} , maximum instructions and data than can be stored is {\bf 256}.
           Address length required is 8 bits.
           \subsection{I/O of RAM}
\end{document}
